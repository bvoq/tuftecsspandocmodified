% Autogenerated translation of index.md by Texpad
% To stop this file being overwritten during the typeset process, please move or remove this header

\documentclass[12pt]{book}
\usepackage{graphicx}
\usepackage{fontspec}
\usepackage[utf8]{inputenc}
\usepackage[a4paper,left=.5in,right=.5in,top=.3in,bottom=0.3in]{geometry}
\setlength\parindent{0pt}
\setlength{\parskip}{\baselineskip}
\setmainfont{Helvetica Neue}
\usepackage{hyperref}
\pagestyle{plain}
\begin{document}

\hrule
title: Tufte Pandoc CSS
subtitle: Starter files for Pandoc Markdown with Tufte CSS
author: Jake Zimmerman

\section*{date: 'November 3, 2016'}

Tufte Pandoc CSS is an attempt to make it as easy as possible to get started
using [Tufte CSS][\textasciicircum{}tufte-css] to write content. It does this by leveraging
Pandoc Markdown's existing features, along with a few new ones implemented as
a \href{http://pandoc.org/scripting.html#json-filters}{JSON filter}.

[\textasciicircum{}tufte-css]:
  \{-\} If you've never heard of Tufte CSS before, take a second to check it out!

\section*{Sidenotes in Markdown}

\begin{verbatim}
<div class="epigraph">
> Tufte CSS provides tools to style web articles using the ideas demonstrated by
> Edward Tufte's books and handouts. Tufte's style is known for its simplicity,
> extensive use of sidenotes, tight integration of graphics with text, and
> carefully chosen typography.
>
> <footer>
> Dave Liepmann, [Tufte CSS]
> </footer>
</div>

\end{verbatim}

Tufte Pandoc CSS aims to be a set of \emph{starter files} for your next project. What
that means is that it provides a number of CSS files, a Pandoc template, a
Makefile, and more to make it as easy as possible to write Markdown using Tufte
CSS.

The biggest barrier that this project overcomes is that Pandoc Markdown doesn't
support side notes nor margin notes by default. This project adds that
functionality.[\textasciicircum{}pdsn] Here's how you can use them:

[\textasciicircum{}pdsn]: In particular, a separate library called \href{https://github.com/jez/pandoc-sidenote}{\texttt{pandoc-sidenote}} handles
  the transformation of footnotes into sidenotes.

```markdown
... In print, Tufte has used the proprietary Monotype
Bembo[\textasciicircum{}note] font. ...

[\textasciicircum{}note]:
  See Tufte's comment in the Tufte book fonts thread.
```

By default, if you use the normal Pandoc syntax for creating footnotes, they'll
become Tufte CSS-style side notes. To get margin notes (i.e., side notes without
numbers), just include \texttt{\{-\}} at the beginning of the note:

\texttt{$<$figure class="fullwidth"$>$}
```markdown
... If you want a sidenote without footnote-style numberings, then you want a
margin note.[\textasciicircum{}mn] On large screens, ...

[\textasciicircum{}mn]:
  \{-\} This is a margin note. Notice there isn't a number preceding the note.
```
\texttt{$<$/figure$>$}

\section*{Syntax-Highlighted Codeblocks}

Pandoc Markdown is excellent for styling code blocks. Indeed, you've already
seen their effect in this document. You can use any of the methods for creating
syntax highlighted code blocks Pandoc permits. For example:

\textasciitilde{}\textasciitilde{}\textasciitilde{}python

\chapter*{Compute elements of the mandelbrot set}

def mandelbrot(a):
    return reduce(lambda z, \_: z * z + a, range(50), 0)

def step(start, step, iterations):
    return (start + (i * step) for i in range(iterations))

rows = (("*" if abs(mandelbrot(complex(x, y))) $<$ 2 else " "
        for x in step(-2.0, .0315, 80))
        for y in step(1, -.05, 41))
\textasciitilde{}\textasciitilde{}\textasciitilde{}

In this document, you're also seeing the effect of the Solarized color scheme.
You can use any of the color schemes Pandoc ships with by default, or this one.

Another feature Pandoc allows that Tufte Pandoc CSS supports is generating line
numbers to accompany a code sample:

```\{.haskell .numberLines\}
merge []         ys                   = ys
merge xs         []                   = xs
merge xs@(x:xt) ys@(y:yt) $\vert$ x $<$= y    = x : merge xt ys
                          $\vert$ otherwise = y : merge xs yt

split (x:y:zs) = let (xs,ys) = split zs in (x:xs,y:ys)
split [x]      = ([x],[])
split []       = ([],[])

mergeSort []  = []
mergeSort [x] = [x]
mergeSort xs  = let (as,bs) = split xs
                in merge (mergeSort as) (mergeSort bs)
```

\section*{Figures \& Sections}

Unfortunately, HTML figures and sections don't have special Markdown syntax. To
include them in your document, you'll have to use the raw HTML. On the bright
side, this usually ends up being pretty painless.

In particular for sections, if you're satisfied with the top-most headings being
wrapped in \texttt{$<$section$>$} tags, you can use the \texttt{--section-divs} flag to \texttt{pandoc}
to automatically wrap sections in divs. This is already enabled in the Makefile
we ship with this project.

Tufte Pandoc CSS improves support for full-width tables and code blocks. Special
attention has been given to ensure that they're fully responsive at all
viewports, just like normal full-width figures. Here's a sample full-width
table:

\texttt{$<$figure class="fullwidth"$>$}

$\vert$ Talk                                 $\vert$ Speaker        $\vert$ Time                   $\vert$
$\vert$ ----                                 $\vert$ -------        $\vert$ ----                   $\vert$
$\vert$ HTML/CSS Primer                      $\vert$ Scott Krulcik  $\vert$ 1:30 p.m. -- 3:00 p.m. $\vert$
$\vert$ JavaScript Primer                    $\vert$ Jake Zimmerman $\vert$ 3:00 p.m. -- 4:30 p.m. $\vert$
$\vert$ UX Prototyping with Framer.js        $\vert$ Lois Yang      $\vert$ 4:30 p.m. -- 6:00 p.m. $\vert$
$\vert$ Frontend Development with Angular.js $\vert$ Sandra Sajeev  $\vert$ 6:30 p.m. -- 8:00 p.m. $\vert$

\texttt{$<$/figure$>$}

As one last quick note: the original Tufte CSS recommends that you always wrap
images in \texttt{$<$figure$>$} tags for optimal responsiveness and layout. Depending on
your tastes, you \textbf{can} choose to omit this. The differences will only take
effect on mobile, where the width of the image will be slightly different from
what it would be if it were properly wrapped. Try it both ways and see whether
you value the convenience of no wrapping or the proper layout that a \texttt{$<$figure$>$}
provides.

\section*{Installation \& Usage}

As mentioned above, Tufte Pandoc CSS is designed to be a collection of starter
files to help you with your next Markdown project. You can learn what files and
tools are available on the \href{https://github.com/jez/tufte-pandoc-css}{GitHub repository}.

As for usage, you are strongly encouraged to look at the \href{https://raw.githubusercontent.com/jez/tufte-pandoc-css/master/docs/index.md}{source of this
document}. There's also an HTML-to-Markdown port of \href{tufte-md/}{the original
Tufte CSS page} along with the accompanying \href{https://raw.githubusercontent.com/jez/tufte-pandoc-css/master/docs/tufte-md/index.md}{source}.

One goal of this project is to support as many of the features you'd "expect" to
work that are available in Pandoc. If your favorite feature doesn't work, let us
know \href{https://github.com/jez/tufte-pandoc-css/issues}{with an issue}.

\end{document}
